\documentclass{beamer}
% Encodings (to render letters with diacritics and special characters)
\usepackage[utf8]{inputenc}
% Language
\usepackage[english]{babel}

\usetheme{Madrid}
\usecolortheme{default}

\pdfstringdefDisableCommands{
  \def\\{}
  \def\texttt#1{<#1>}
}

\newcommand{\email}[1]{
{\footnotesize \texttt{\href{mailto:#1}{#1}} }
}

\usepackage{caption}
\DeclareCaptionFont{black}{\color{black}}
\DeclareCaptionFormat{listing}{{\tiny \textbf{#1}#2#3}}
\captionsetup[lstlisting]{format=listing,labelfont=black,textfont=black}

\usepackage{listings}
\lstset{
    frame=tb, % draw frame at top and bottom of the code
    basewidth  = {0.5em,0.5em},
    numbers=left, % display line numbers on the left
    showstringspaces=false, % don't mark spaces in strings  
    commentstyle=\color{green}, % comment color
    keywordstyle=\color{blue}, % keyword color
    stringstyle=\color{red}, % string color
	aboveskip=-0.2em,
    belowskip=-0.2em,
    basicstyle=\tiny
}
\lstset{literate=
  {á}{{\'a}}1 {é}{{\'e}}1 {í}{{\'i}}1 {ó}{{\'o}}1 {ú}{{\'u}}1
  {Á}{{\'A}}1 {É}{{\'E}}1 {Í}{{\'I}}1 {Ó}{{\'O}}1 {Ú}{{\'U}}1
  {à}{{\`a}}1 {è}{{\`e}}1 {ì}{{\`i}}1 {ò}{{\`o}}1 {ù}{{\`u}}1
  {À}{{\`A}}1 {È}{{\'E}}1 {Ì}{{\`I}}1 {Ò}{{\`O}}1 {Ù}{{\`U}}1
  {ä}{{\"a}}1 {ë}{{\"e}}1 {ï}{{\"i}}1 {ö}{{\"o}}1 {ü}{{\"u}}1
  {Ä}{{\"A}}1 {Ë}{{\"E}}1 {Ï}{{\"I}}1 {Ö}{{\"O}}1 {Ü}{{\"U}}1
  {â}{{\^a}}1 {ê}{{\^e}}1 {î}{{\^i}}1 {ô}{{\^o}}1 {û}{{\^u}}1
  {Â}{{\^A}}1 {Ê}{{\^E}}1 {Î}{{\^I}}1 {Ô}{{\^O}}1 {Û}{{\^U}}1
  {Ã}{{\~A}}1 {ã}{{\~a}}1 {Õ}{{\~O}}1 {õ}{{\~o}}1
  {œ}{{\oe}}1 {Œ}{{\OE}}1 {æ}{{\ae}}1 {Æ}{{\AE}}1 {ß}{{\ss}}1
  {ű}{{\H{u}}}1 {Ű}{{\H{U}}}1 {ő}{{\H{o}}}1 {Ő}{{\H{O}}}1
  {ç}{{\c c}}1 {Ç}{{\c C}}1 {ø}{{\o}}1 {å}{{\r a}}1 {Å}{{\r A}}1
  {€}{{\euro}}1 {£}{{\pounds}}1 {«}{{\guillemotleft}}1
  {»}{{\guillemotright}}1 {ñ}{{\~n}}1 {Ñ}{{\~N}}1 {¿}{{?`}}1
}

\usepackage{dirtree}

\usepackage[style=british]{csquotes}

\usepackage{tabularx}

\usepackage{graphicx}
	\graphicspath{{./images/}{../documentacao/}}
 
%Information to be included in the title page:
\AtBeginDocument{
\title[Ball Sort Puzzle (Checkpoint)]{Ball Sort Puzzle (Checkpoint)}
\author[Group 48]{
\begin{tabular}{r l}
	\email{up201806581@fe.up.pt} & Bernardo António Magalhães Ferreira \\
	\email{up201806429@fe.up.pt} & Diogo Miguel Ferreira Rodrigues     \\
	\email{up201806330@fe.up.pt} & Rafael Soares Ribeiro
\end{tabular}
}
\institute[FEUP/AEDA]{Faculdade de Engenharia da Universidade do Porto \\ Artificial Intelligence (IART) -- Group 48}
\date[?/?/2021]{?th of ?, 2021}
}

\begin{document}
\frame{\titlepage}

\begin{frame}
\frametitle{Work specification}

This work has the goal to solve a solitaire game called Ball Sort Puzzle made by Spica Game Studio using Heuristic Search Methods.

The Ball Sort Puzzle consists in sorting a set of balls by color in different tubes. This game's board is constituted of a set of tubes and balls with different colors. The number of tubes is greater than or equal to the set of balls with the same color. All these balls are randomnly distributed in the tubes. One important tubes' characteristic is that their height has a limit. The goal is to put all balls with the same color in only one tube. 

To solve this problem, first of all we need to implement the game for a human player. We will use a graphic design to allow us to see the 
game's evolution and give the player a way to interact with the board. Second of all we will implement several search algorithms that will allow a computer to solve the game. We can use uninformed search algorithms such as breadth-first search, depth-first search, iterative deepening and uniform cost or heuristic search algorithms like greedy search and A* algorithm. 

This search algorithms are used to search in a tree of game states the game's goal and the path to reach it.
\end{frame}

\begin{frame}
\frametitle{Related work from other authors}

Sample text.

\end{frame}

\begin{frame}
\frametitle{Formulation}

Sample text.

\end{frame}

\begin{frame}
\frametitle{Implementation (so far)}

\end{frame}

\end{document}
